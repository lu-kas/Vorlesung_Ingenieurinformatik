\documentclass[convert={density=300,size=1920x1080,outext=.png}]{standalone}
\usepackage{tikz,pgfplots}
\usepackage{xcolor}

\renewcommand{\familydefault}{\sfdefault}

\usepackage{sansmath} % sans serif math
\sansmath % if you use it globaly

\usetikzlibrary{decorations.pathreplacing}

\begin{document}

	\definecolor{border_color}{RGB}{80,116,172}
	\definecolor{fill_color}{RGB}{162,203,240}
	\definecolor{dot_color}{RGB}{60,60,60}

    \begin{tikzpicture}[scale=3]

        \tikzstyle{every node}=[font=\fontsize{35}{0}\selectfont]
        \tikzstyle{boxes_style}=[draw=border_color,fill=fill_color, line width=1.5mm]
        
        \draw node[above] at (-0.5, 1.25) {Bit}; 
        
        \foreach \x in {0,...,7} {
            \pgfmathtruncatemacro\label{7-\x};
            \draw[boxes_style] (\x,0) rectangle (\x+1,1); \draw node[above] at (\x+0.5, 1.25) {\label};   
        }
    
        \def\yshift{-0.05}
        \draw[line width=1mm,decorate,decoration={brace,amplitude=22pt,mirror},yshift=0pt] (1,\yshift) -- (4,\yshift) node [black,midway,yshift=-2.0cm,below] {Bin\"are Darstellung des Exponenten $e$};
        \draw[line width=1mm,decorate,decoration={brace,amplitude=22pt,mirror},yshift=0pt] (4,\yshift) -- (8,\yshift) node [black,midway,yshift=-0.8cm,below] {Bin\"are Darstellung der Mantisse $m$};
        
        \draw[line width=1mm] (0.5,\yshift) -- (0.5,\yshift-0.25) node[below] {Vorzeichen $s$};
    \end{tikzpicture}
\end{document}