\documentclass[tikz, convert={density=300,size=1920x1080,outext=.png}, margin=1mm]{standalone}
%\documentclass{standalone}
\usepackage{tikz,pgfplots}
\usepackage{xcolor}

\usepackage[ngerman]{babel}

\renewcommand{\familydefault}{\sfdefault}


\usepackage{sansmath} % sans serif math
\sansmath % if you use it globaly

\usetikzlibrary{mindmap, shadows}
\usetikzlibrary{decorations.pathreplacing}


\begin{document}

	\definecolor{my_blue}{RGB}{80,116,172}
	\definecolor{my_blue_light}{RGB}{163,203,240}
	\definecolor{my_orange}{RGB}{217,132,92}
	\definecolor{my_orange_light}{RGB}{251,180,139}
	\definecolor{my_green}{RGB}{91,166,108}
	\definecolor{my_green_light}{RGB}{148,228,167}
	\definecolor{my_red}{RGB}{192,80,86}
	\definecolor{my_red_light}{RGB}{251,159,158}	
	\definecolor{my_gray_dark}{RGB}{60,60,60}

    \begin{tikzpicture}[mindmap]
  		\begin{scope}[every node/.style={concept, circular drop shadow,
  										 execute at begin node=\hskip0pt}, 
  					  root concept/.append style={concept color=my_gray_dark, fill=white, 
  					  							  line width=1ex, text=my_gray_dark,
  					  							  font=\Large\scshape\sf},
  					  T1/.style={concept color=my_blue_light, fill=black}, 
  					  T2/.style={concept color=my_orange_light}, 
  					  T3/.style={concept color=my_green_light}, 
  					  T4/.style={concept color=my_red_light}, 
  					  grow cyclic, 
  					  level 1/.append style={level distance=4.5cm, fill=white, 
  					  					     sibling angle=90, font=\large\sf},
  					  level 2/.append style={level distance=3cm, fill=my_blue_light, 
  					  					     sibling angle=55,font=\small\sf}] 
\node [root concept] {Ingenieurinformatik} % root
	  child [T1] { node {Numerik} 
	  	child { node {Ableitungen} } 
	  	child { node {Integrale} } 
	  	child { node {ODE} } 
	  	child { node {PDE} } 
	  }
	  child [T2] { node {Datenanalyse} 
	  	child { node {Optimierung} } 
	  	child { node {Analyse} } 
	  	child { node {Visualisierung} } 
	  	child { node {Einlesen} }
	  }
	  child [T3] { node {Python} 
	  	child { node {Funktionen} } 
	  	child { node {Flusskontrolle} }
	  	child { node {JupyterLab} } 
	  	child { node {Grundlagen} } 	  	
	  }
      child [T4] { node {Computer}
        child { node {Digitalisierung} }
        child { node {Algorithmen} }
        child { node {Hardware} }
        child { node {Software} }
      }
	  ;

  		\end{scope}
	\end{tikzpicture}

\end{document}